\subsection{Sudoku}

Everyone is familiar with Sudoku puzzles, which appear in newspapers daily the world over. A typical puzzle is shown in Figure \ref{sudokuexampleone}. To complete the puzzle requires the puzzler to fill the empty cells with numbers 1, \ldots , 9 in such a way as to have exactly one of every number in every row, every column and every of the small 3 by 3 boxes.

\begin{figure}[h]
\label{sudokuexampleone}
\centering
  \sudokuexampleone
\caption{A Sudoku puzzle}
\end{figure}

A well-formed Sudoku puzzle has a unique solution. This means that the puzzle can be solved by logic alone, without any guessing.

Sudoku puzzles have a variety of different difficulty levels. Harder puzzles typically have fewer prescribed symbols. It is unknown to this day how few cells need to be filled for a Sudoku puzzle to have a unique solution. Well-formed Sudoku with 17 symbols exist. It is unknown whether or not there exists a well-formed puzzle with 16 clues.

\subsection{\libname}

The authors have written an open-source library for modeling Sudoku puzzles in a variety of different mathematical domains. The source-code for \libname\ is available at \libwebsite.

Cells in the Sudoku puzzle are represented by integers. The cell in row $i$ and column $j$ of a puzzle of dimension (\texttt{boxsize}) $n$ with $r$ rows is represented by the integer $(i - 1)r + j$. Standard puzzles have $r = n^2$ rows so the integer is $(i - 1)n^2 + j$. 

\begin{figure}[h]
\label{sudokulabeling}
\centering
  \sudokulabeling
\caption{Cell labeling of Sudoku puzzle of \texttt{boxsize} 3.}
\end{figure}

Sudoku puzzles are represented, in \libname, as dictionaries. So, for example, the puzzle shown in Figure \ref{sudokuexampleone} is represented by the dictionary:

\begin{lstlisting}[caption=Sudoku puzzle dictionary,label=puzzledict]
{1: 2, 2: 5, 5: 3, 7: 9, 9: 1, 
11: 1, 15: 4, 19: 4, 21: 7, 25: 2, 
27: 8, 30: 5, 31: 2, 41: 9, 42: 8, 
43: 1, 47: 4, 51: 3, 58: 3, 59: 6, 
62: 7, 63: 2, 65: 7, 72: 3, 73: 9, 
75: 3, 79: 6, 81: 4}
\end{lstlisting}

In practice, the user mainly interacts with \texttt{sudoku.py} either by creating specific puzzles instances through input of puzzle strings, directly or from a text file, or by using generator functions. 

For example, the puzzle dictionary in Figure \ref{puzzledict} can be built from a puzzle string through use of the \texttt{sudoku\_string\_to\_dict} function.

\lstinputlisting[firstline=1,lastline=44,caption=Building a Sudoku dictionary,label=stringtodict]{../../demos/article_demo.py}

Or a random puzzle can be built by using the XXX function.

XXX random puzzle demo listing XXX

Simple functions are provided to access certain parameters associated with a puzzle.

\lstinputlisting[firstline=15,lastline=20,caption=Code example,label=simplefunctions]{../../src/sudoku.py}

The main power behind \libname, however, is the modeling capability of the library. In the next section we introduce the different modeling concepts and show how to use existing Python components to build models of Sudoku puzzles. 

