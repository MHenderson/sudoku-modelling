Everyone is familiar with Sudoku puzzles, which appear in newspapers daily the world over. A typical puzzle is shown in Figure \ref{sudokuexampleone}. To complete the puzzle requires the puzzler to fill the empty cells with numbers 1, \ldots , 9 in such a way as to have exactly one of every number in every row, every column and every of the small 3 by 3 boxes.

\begin{figure}[h]
\label{sudokuexampleone}
\centering
  \sudokuexampleone
\caption{A Sudoku puzzle}
\end{figure}

A well-formed Sudoku puzzle has a unique solution. This means that the puzzle can be solved by logic alone, without any guessing.

Sudoku puzzles have a variety of different difficulty levels. Harder puzzles typically have fewer prescribed symbols. It is unknown to this day how few cells need to be filled for a Sudoku puzzle to have a unique solution. Well-formed Sudoku with 17 symbols exist. It is unknown whether or not there exists a well-formed puzzle with 16 clues.

In \texttt{sudoku.py}, Sudoku puzzles are represented, internally, as dictionaries. Cells in the Sudoku puzzle are represented by integers. The cell in row $i$ and column $j$ of a puzzle of dimension (\texttt{boxsize}) $n$ with $r$ rows is represented by the integer $(i - 1)r + j$. Standard puzzles have $r = n^2$ rows so the integer is $(i - 1)n^2 + j$. 

So, for example, the puzzle shown in Figure \ref{sudokuexampleone} is represented by the dictionary:

\begin{lstlisting}[caption=Sudoku puzzle dictionary,label=puzzledict]
{1: 2, 2: 5, 5: 3, 7: 9, 9: 1, 
11: 1, 15: 4, 19: 4, 21: 7, 25: 2, 
27: 8, 30: 5, 31: 2, 41: 9, 42: 8, 
43: 1, 47: 4, 51: 3, 58: 3, 59: 6, 
62: 7, 63: 2, 65: 7, 72: 3, 73: 9, 
75: 3, 79: 6, 81: 4}
\end{lstlisting}

However, the user mainly interacts with \texttt{sudoku.py} through input and output of puzzle strings. For example, the puzzle dictionary in Figure \ref{puzzledict} can be built from a puzzle string through use of the \texttt{sudoku\_string\_to\_dict} function.

\begin{lstlisting}[caption=Building a Sudoku dictionary,label=stringtodict]
In [1]: import sudoku

In [2]: p = """
   ...: 25..3.9.1
   ...: .1...4...
   ...: 4.7...2.8
   ...: ..52.....
   ...: ....981..
   ...: .4...3...
   ...: ...36..72
   ...: .7......3
   ...: 9.3...6.4
   ...: """
In [52]: sudoku.sudoku_string_to_dict(p)
Out[52]: 
{1: 2,
 2: 5,
 5: 3,
 7: 9,
 9: 1,
 11: 1,
 15: 4,
 19: 4,
 21: 7,
 25: 2,
 27: 8,
 30: 5,
 31: 2,
 41: 9,
 42: 8,
 43: 1,
 47: 4,
 51: 3,
 58: 3,
 59: 6,
 62: 7,
 63: 2,
 65: 7,
 72: 3,
 73: 9,
 75: 3,
 79: 6,
 81: 4}

\end{lstlisting}

Simple functions are provided to access certain parameters associated with a puzzle.

\lstinputlisting[firstline=15,lastline=20,caption=Code example,label=simplefunctions]{../src/sudoku.py}

