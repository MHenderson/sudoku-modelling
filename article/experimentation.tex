In this section we demonstrate how to use \texttt{sudoku.py} to create experimentation scripts. For the purposes of demonstration, we reproduce several results from the literature. We show how to enumerate Shidoku puzzles, how to color the Sudoku graph with the minimal number of colors, how to investigate minimally uniquely completable Sudoku puzzles, how to investigate phase transition phenomena in randomly generated Sudoku puzzles. Finally, we look at a competition, closely related to Sudoku puzzles, which was held by Mathworks in 2005. 

The intention of this section is to show how \texttt{sudoku.py} makes the task of writing these experimental investigation scripts very easy.

\subsection{Enumerating Shidoku}
\label{sec:shidoku}

\lstinputlisting[firstline=77, lastline=91caption=Enumerating Shidoku via constraint model]{../demos/article_demo.py}

\subsection{Coloring the Sudoku graph}
\label{sec:coloring}

\subsection{Minimally uniquely completable puzzles}
\label{sec:minimal}

\subsection{Phase transition phenomena with random puzzles}
\label{sec:phase}

\cite{lewismetaheur}

\subsection{The Matlab Sudoku contest}
\label{sec:contest}

\url{http://www.mathworks.com/contest/sudoku/rules.html}

